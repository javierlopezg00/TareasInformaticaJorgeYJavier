\documentclass[11pt,a4paper]{article}
\usepackage[latin1]{inputenc}
\usepackage[spanish]{babel}
\usepackage{amsmath}
\usepackage{amsfonts}
\usepackage{amssymb}
\usepackage{graphicx}
\usepackage[left=2cm,right=2cm,top=2cm,bottom=2cm]{geometry}
\author{Javier Lopez Y Jorge Gerrero }
\title{Hoja de trabajo 1}
\begin{document}
\begin{Huge}
Hoja de trabajo \# 1
\end{Huge}

\begin{itemize}
\item Javier Lopez 2019-1204
\item Jorge Guerrero  2019-1096
\end{itemize}

\section*{Ejercicio \#1 (5\%)}
{\Large Nuestro repositorio es: \\https://github.com/javierlopezg00/TareasInformaticaJorgeYJavier.git}
\section*{Ejercicio \#2: Abstracci\'on (50\%)}

\section*{1) Conjunto de nodos}

\begin{LARGE}
\begin{flushleft}
(1,1)|(2,2)|(3,3)|(4,4)|(5,5)|(6,6)\\
(1,2)|(2,3)|(3,4)|(4,5)|(5,6)\\
(1,3)|(2,4)|(3,5)|(4,6)\\
(1,4)|(2,5)|(3,6)\\
(1,5)|(2,6)\\
(1,6)
\end{flushleft}
\end{LARGE}
\section*{2) Conjunto de v�rtices}
\begin{LARGE}
\begin{flushleft}
(1,1)=  (1,2)(1,3)(1,4)(1,5)\\
(2,2)=  (2,1)(2,3)(2,4)(2,6)\\
(3,3)=  (3,1)(3,2)(3,5)(3,6)\\
(4,4)=  (4,1)(4,2)(4,5)(4,6)\\
(5,5)=  (5,1)(5,3)(5,4)(5,6)\\
(6,6)=  (6,2)(6,3)(6,4)(6,5)\\

\end{flushleft}
\begin{flushleft}
(1,2)	=	(1,1)(2,2)  (1,3)(1,4)(1,6)  (3,2)(4,2)(6,2)\\
(1,3)	=	(1,1)(3,3)  (1,2)(1,5)(1,6)  (2,3)(5,3)(6,3)\\
(1,4)	=	(1,1)(4,4)  (1,2)(1,5)(1,6)  (2,4)(5,4)(6,4)\\
(1,5)	=	(1,1)(5,5)  (1,3)(1,4)(1,6)  (3,5)(4,5)(6,5)\\
(1,6)	=	(1,2)(1,3)  (1,4)(1,5)(2,6)  (3,6)(4,6)(5,6)\\
\end{flushleft}

\begin{flushleft}
(2,3)= (2,2)(3,3)  (2,1)(2,5)(2,6)  (1,3)(4,3)(6,3)\\
(2,4)= (2,2)(4,4)  (2,1)(2,5)(2,6)  (1,4)(3,4)(6,4)\\
(2,5)= (2,1)(2,3)  (2,4)(2,6)(1,5) (3,5)(4,5)(6,5)\\
(2,6)= (2,2)(6,6)  (2,3)(2,4)(2,5)  (1,6)(3,6)(4,6)\\
\end{flushleft}

\begin{flushleft}
(3,4)=  (3,1)(3,2)  (3,5)(3,6)(1,4)   (2,4)(5,4)(6,4)\\
(3,5)=  (3,3)(5,5)  (3,1)(3,4)(3,6)   (1,5)(4,5)(6,5)\\
(3,6)=  (3,3)(6,6)  (3,2)(3,4)(3,5)   (1,6)(2,6)(5,6)\\
\end{flushleft}

\begin{flushleft}
(4,5)=  (4,4)(5,5)  (4,1)(4,3)(4,6)   (1,5)(2,5)(6,5)\\
(4,6)=  (4,4)(6,6)  (4,1)(4,2)(4,5)   (1,6)(2,6)(5,6)\\
\end{flushleft}

\begin{flushleft}
(5,6)=  (5,5)(6,6)  (5,2)(5,3)(5,4)   (1,6)(3,6)(4,6)\\
\end{flushleft}
\end{LARGE}
\section*{3) Preguntas}

\begin{Large}
\begin{itemize}
 \item �Que estructura de datos podr�a representar un lanzamiento de dados?
 \begin{itemize}
 \item El lanzamiento de los dados es una secuencia que forma v�rtices entre el estado anterior y el estado actual que cumplen con las propiedad regida por la rotaci�n del dado la cual es solo puede girar a un lado adyacente.
 \end{itemize}

\item �Que algoritmo  utilizar para generar dicha estructura?
\begin{flushleft}
\textbf{Algoritmo\\}
1)Inicio.\\
2)Ambos dados empiezan en un numero entre el 1 y el 6. \\
3)Solo un dado rota noventa grados de forma aleatoria y el otro permanece como esta.\\
4)El dado que se movi� muestra un numero aleatorio entre los n�meros posibles gracias al giro de noventa grados.\\
5)El segundo dado repite el paso 3 y 4.\\
6)Ambos dados dar�n un resultado.\\
7)Fin\\

\end{flushleft}
\item �Como nos aseguramos que ese algoritmo siempre produce un resultado?
\begin{flushleft}
\begin{enumerate}
\item Se tiene que contar con dos dados
\item No es necesario hacer la rotaci�n de dados ya que siempre habr� un resultado.
\item Los pasos del algoritmo anterior se tienen que cumplir.
\item El resultado mostrado por los dados tiene que aparecer en el grafo creado anteriormente.
\end{enumerate}
Si se cumple todo lo mencionado anteriormente se obtendr� una pareja de n�meros que representan la combinaci�n de resultados de ambos dado.
\end{flushleft}
\end{itemize}

\end{Large}
\end{document}