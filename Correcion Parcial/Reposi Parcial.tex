\documentclass[11pt,a4paper]{article}
\usepackage[latin1]{inputenc}
\usepackage[spanish]{babel}
\usepackage{amsmath}
\usepackage{amsfonts}
\usepackage{amssymb}
\usepackage{graphicx}
\usepackage[left=2cm,right=2cm,top=2cm,bottom=2cm]{geometry}
\author{Javier Lopez Y Jorge Gerrero }
\title{Hoja de trabajo 1}
\begin{document}
\begin{Huge}
Hoja de trabajo \# 3
\end{Huge}
\section*{Ejercicio \#1}
$
1) \forall n \geq 1 \\
2 \times n es par \\
\textbf{Caso base} \\
n = 1 \\
2(1)$ es par \\$
2$ es par \\$
\textbf{Caso inductivo} \\
2 \ast n $ es par \\$
n = n+1\\
2(n+1) $es par \\
2n + 2 es par \\
2 es par \\
2) $\forall \geq 4 \\
2^{n} < n! \\
n! = 1x2x3x...x(x-1)xn$\\
\textbf{Caso Base}\\
$n = 4\\
2^{4}<1\times2\times3 \\
16<2^{4}$ \\
\textbf{Caso inductivo} \\
$n = n+1 \\
\forall n \geq 4, 2^{n} < n! \\
2^{n}\ast 2 < n! (n+1) \\
2[2^{n}<n!(n+1)] \\
$
\section*{Ejercicio \#2}
$1) n! = 1\otimes 2 \otimes 3 \otimes ... \otimes (n+1) \otimes n\\ $
$ n! $
$\left\{
	\begin{array}{ll}
		
		1, 5: n = 0 \\
		(x!) \otimes (\sigma (x))		
		
	\end{array}
\right.$ \\ \\
2) $ a \ominus b = $
$\left\{
	\begin{array}{ll}
		a   \mbox{ / }  b=0 \\
		0 \mbox{ / }    a\wedge b = 0 \\
		\sigma (x\ominus b) \mbox{/} a = \sigma (x)
	\end{array}
\right.$ \\
$4) a^{b} = b\otimes a\otimes a ... ($b veces)$\\ $
$ a^{b}$
$\left\{
	\begin{array}{ll}
		1 "si" b =0 \\
		0 "si" a = 0\\
		a\otimes a $"si"$ b= \sigma (i)
	\end{array}
\right.$ \\
\section*{Ejercicio \#3}
$ a \otimes b = $
$\left\{
	\begin{array}{ll}
		0   \mbox{ / }  a\vee b=0 \\
		a  \mbox{ / }  b=1 \\  
		B \mbox{ / }  a=1 \\
		a\oplus(a\otimes x) \mbox{ / }  \sigma (x)
	\end{array}
\right.
\\
2 \otimes a = a\oplus a \\
$
\textbf{Caso base}\\
$
2 \otimes 0 = 0 \oplus 0 \\
0 \oplus (0 \otimes 1) = 0 \oplus 0 \\
0 \oplus 0 = 0 \oplus 0 \\
0 = 0\\
$
\textbf{Caso inductivo }\\
$
2 \otimes \sigma (a) = \sigma (a) \oplus \sigma (a) \\ 
\sigma (a) \oplus (\sigma (a) \otimes 1) = \sigma (a) \oplus \sigma (a)\\
\sigma (a) \oplus (\sigma (a)) = \sigma (a) \oplus \sigma (a)\\
\sigma (a) \oplus \sigma (a) = \sigma (a) \oplus \sigma (a)\\
\sigma (\sigma (a \oplus a)) = \sigma (\sigma (a \oplus a)) \\
(a \oplus a) = (a \oplus a) \\
(a \oplus a) = (2 \otimes a)
$
\end{document}
