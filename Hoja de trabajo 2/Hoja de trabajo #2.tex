\documentclass[11pt,a4paper]{article}
\usepackage[latin1]{inputenc}
\usepackage[spanish]{babel}
\usepackage{amsmath}
\usepackage{amsfonts}
\usepackage{amssymb}
\usepackage{graphicx}
\usepackage[left=2cm,right=2cm,top=2cm,bottom=2cm]{geometry}
\author{Javier Lopez Y Jorge Gerrero }
\title{Hoja de trabajo 1}
\begin{document}

\section*{Ejercicio \#1 (50\%)}

\begin{flushleft}
Demostrar utilizando inducci\'on:
\[
        \forall\ n.\ n^3\geq n^2
\]
\\donde $n\in\mathbb{N}$\\
\textbf{Caso base}
\\n^3 \geq n^2\\
n = 0\\
 (0)*(0)*(0) = (0)*(0)\\
 \\
\textbf{Caso inductivo}
\\ n^3 \geq n^2\\
 
 
 (n+1)^3 \geq(n+1)^2\\
 (n+1)(n+1)(n+1) \geq (n+1)(n+1)\\
 (n+1) * (n*n+2n+1*1) \geq (n*n+2n+1*1)\\
 
 u =  (n*n+2n+1*1)\\
 
 (n+1) * u \geq u\\
 n*u + u*1 \geq u\\
 (n+1) \geq 1\\
 n \geq 0\\
 \\ \\
 \section*{Ejercicio \#2 (50\%)}
 Demostrar utilizando inducci\'on la desigualdad de Bernoulli:
\[
        \forall\ n.\ (1+x)^n\geq nx
\]
\\donde $n\in \mathbb{N}$, $x\in \mathbb{Q}$ y $x\geq -1$\\ \\
{\large  \textbf{Caso base}}\\
 (x+1)^n >= n*x+1\\
 n= 0\\
 (x+1)^n >= n*x+1\\
 (x+1)^(0) >= (0)*x\\
 0*x + 0*1 = 0*x\\
 0 = 0\\
  
\textbf{  Caso inductivo}\\
  (x+1)^(n+1) >= (n+1)x+1\\
  (x+1)^n+1 >= (n+1)*x+1\\
  
  (x+1)^n * (x+1) >= (n+1)x+1\\
  (x+1)^n * (x+1) >= nx+x+1\\
  
  (x+1)^n+1 >= x(n+1)+1\\
  
  (n+1) = u\\
  (x+1)^u >= xu+1\\
\end{flushleft}
 

\end{document}